\documentclass[a4paper,12pt]{article}

\usepackage[utf8]{inputenc}
\usepackage{graphicx}
\usepackage{mathtools}

\title{Math!}
\author{Rodrigo Caetano Rocha}
\date{\today}

\begin{document}
\maketitle

Exemplo de uma fórmula inline: $E=mc^2$ representa uma equivalência entre massa e enegia.

\[
\binom{n}{k} = \frac{n!}{k!(n-k)!}
\]

\[
  \left(\frac{x^2}{y}\right)
\]
\[
   x\in \{ n | 0 < n \leq 100 \}
\]

\[
   A \cap B \neq \emptyset
\]

\[
   A \cup B
\]
\[
   A \setminus B
\]
\[
   A \subset B
\]
\[
   A \subseteq B
\]
\[
   A \supset B
\]
\[
   A \supseteq B
\]
\[
M_{3\times 3} = 
\begin{pmatrix}
1 & 2 & 3\\
4 & 5 & 6\\
7 & 8 & 9
\end{pmatrix}
\]

\[
\lim_{n \to \infty} \frac{1}{n} = 0
\]

\end{document}
